\chapter{Implementáció}
\label{chapImplemetacio}

%Install ubuntu server
%Install docker
%https://docs.docker.com/install/linux/docker-ce/ubuntu/
%Install docker registry
%Install docker registry web interface
%https://hub.docker.com/r/hyper/docker-registry-web/
%Install Jekins docker
%https://hub.docker.com/r/jenkins/jenkins/
%Install Jenkins docker plugin
%Docker Slaves plugin
\section{Szerverek}

\subsubsection{Safranek}

\paragraph{}
A feladatok közül a legelső a már meglévő Safranek szerver frissítése és konfigurálása volt. 
A Safraneken egy Ubuntu server 16.04-es verzió futott, így nem volt szükség sem az operációs rendszer, sem pedig a disztribúció cseréjére, mivel a projektet az Ubuntu szerver ezen verziója tökéletesen kiszolgálja.
A frissítéseket követően, a Jenkins CI futtatását kellett előkészíteni. 
Egy jenkins nevű felhasználót kellett létrehozni, hogy szeparáltan lehessen futtatni a programokat és az autentikáció során ne a root felhasználó adatait keljen felhasználni. 

Mivel a TomCat és a Jenkins is egy-egy java alapú alkalmazás ezért a Java JRE és JDK legfrissebb verzióját kellett felkellet telepíteni. 
Azt a következő két parancs alkalmazásával: 

----"sudo apt-get install default-jre \&\& sudo apt-get install default-jdk"------


\paragraph{}
Az első felhasználó létrehozásához és a komponensek telepítése előtt egy a "/.jenkins/secrets/initialAdminPassword" fájlban található jelszót kell bemásolni a formba a folytatáshoz. 
Miután ez megtörtént a Jenkins CI admin felülete tárul elénk \ref{fig:clearjenkins}. 

\subparagraph{Jenkins}

A Jenkin CI 